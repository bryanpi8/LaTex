\section{Timeline}
A timeline for this semester can be seen in figure 3 there are a significant amount of tasks that overlap. In some cases it could be seen that there are also very few tasks for november. Both of these observations are due to the fact that the PicoBalloon is highly affected by weather and the fact that future work on the PicoBalloon is based on how the second test goes in october, weather prevailing.
\begin{figure}[H]
    \begin{center}
            \includegraphics[width=1\textwidth]{Images/SSDS Timeline.jpg}\caption{A (mostly) semester long timeline for SSDS work}
    \end{center}
\end{figure}
The DeSCENT mission tasks are all in orange, these tasks are on a rolling timeline essentially as some of them are sequential, and V2 is not the main priority. The launch window could open as soon as november, so working out any issues with the V1 is crucial  before then. If the focus this semester becomes solely V1 then V2 will be the focus of next semester. Figure 3 is difficult to read so here is the data that went into the chart
\begin{table}[!htp]\centering
    \scriptsize
    \begin{tabular}{lrr}\toprule
        Task &Duration(days) \\\cmidrule{1-2}
        GPS Issue verification &5 \\
        Solar Testing &7 \\
        Long Duration Solar Testing &7 \\
        Long Duration Balloon Test &14 \\
        Design and Results Review &3 \\
        WIO-E5 breadboard setup &21 \\
        Breadboard Testing &21 \\
        PCB Layout &14 \\
        Breadboard Troubleshooting &21 \\
        PCB Design Review &3 \\
        PCB testing &14 \\
        \bottomrule
    \end{tabular}
\end{table}