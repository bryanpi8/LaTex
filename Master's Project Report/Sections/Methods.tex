\section{Methods}
During the course of this project I have been able to learn a significant amount about antenna tuning, new programming methods, and how the multilpe agencies that control spaceflight interact during the development phase of a launch.
First let's explore what progress has been made on the alpha chipsat as that was my main focus for the majority of the semester so far. A short list of whats been accomplished on alpha so far is:
\begin{enumerate}
    \item Solar Testing
        \begin{enumerate}
        \item Power regulation circuit capacitors
        \item Solar power characterization for the Balloon test
        \end{enumerate}
    \item v4.76 Chipsat assembly
        \begin{enumerate}
        \item Soldered all components
        \item Tested functionality and troubleshooted
        \item Discovered a major flaw with GPS and clock speed
        \end{enumerate}
    \item LightSail Antenna tuning
        \begin{enumerate}
        \item Assisted in tuning antennas on one Chipsat for the alpha lightsail
        \end{enumerate}
    \item GPS lock Testing
\end{enumerate}

Working with the solar system on the alpha chipsat was fairly straightforward, we have lights in the basement that match the solar panels peak frequency of around 850nm. The lights are setup to energize the solar panel with a short circuit current of around 100mA. This current should be ample to power the Chipsat throuhg boot up, though issues may occur on a transmission.
One round of this testing was done in Rhodes B40, using a specific light to make sure that the chipsats would function on solar power exclusively. This test showed overall very bad results, the EDU chipsat 
\textbf{INSERT PICTURE HERE}
